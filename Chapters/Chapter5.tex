\chapter{Conclusions and future works} % Main chapter title

\label{Chapter5} % For referencing the chapter elsewhere, use \ref{Chapter3} 

%----------------------------------------------------------------------------------------
This work focused on finding a suitable method for performing acoustic contrast control and evaluated its performances and limitations, both inherent to the algorithm itself or given by external factor such as the surrounding environment.
\\
The experimental conditions in the anechoic chamber have been defined, by testing the assumptions regarding the non-linear distortions generated by the loudspeakers, as well as the background noise and the free field attenuation, together with the limit introduced by spatial and frequency aliasing. Once the operating environment has been treated, it was possible to analyze the performances of the algorithm at boundary conditions to find limitations, edge cases and opportunities for future improvements.
\\
Successively, a small evolution of the algorithm has been proposed, it tried to tackle the uncertainties introduced by a reflector surface given by the resulting increase in the energy content inside the controlled zones, caused by the redirected soundwaves.
\\
The idea behind the modification proposed was that, if we could separate the contribution given by the direct hit and the reflected hit we could calculate two contrast filters separately, therefore speeding up the calculations with little change to the contrast figure.
\\
The algorithm proposed has limited applicability in a scenario where the reflections can be easily identified in the impulse response and subsequently windowed out. There are still limitations on this approach given by the distortion that the windowing function introduces in the impulse response. Also the analysis of the reflections, even though shows promising results, is limited to a single scenario and provides results that cannot be considered conclusive, mostly due to the poor choice of the reflector used.
\\
Finally the ACC method developed has been tested in a real environment, demonstrating that some acoustic contrast is possible in an highly reflective environment, but its performances are still not satisfying enough to be applied in a day-to-day scenario.
\\
\\
From the experiments some areas of possible improvements can be identified, specifically two modifications show some potential for an improved, future, version of the current algorithm:

\begin{itemize}
\item Introducing single regularization terms ($\beta, \delta$) for each loudspeaker.
\item Limiting the distortion on the frequency response of the system by limiting the difference in SPL between lower and upper frequency limit, in other words, limiting the tilt of the reproduced signal towards the higher frequency bands.
\end{itemize}

Introducing single regularization terms distributes the control effort more evenly towards all the speakers, generating a more pleasant experience for a potential listener.
\\
The predilection of the BACC-RD algorithm towards the higher frequencies is a problem already observed by \parencite{schellekens_time_2016} which proposes a modification (called BACC-RTE), which limits the mean variation between two adjacent control frequencies by introducing an additional constraint to the maximization problem \ref{eqn:optimization}, thus limiting the skewing effect. The performances of this algorithm look promising and together with the introduction of individual regularization might finally provide satisfying results even in an highly reflective environment.
\\
This is because the reflections introduces standing wave effects. 
These are highly spatially sensitive phenomena, meaning that even a small variation in the position of the speakers might eliminate this phenomena, this can be done by simply penalizing the control effort provided by a speaker (therefore limiting its output) to the advantage of another situated in a more convenient position.
\\
\\
This kind of improvements is left as a suggestion for future works that might follow.